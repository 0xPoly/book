\newacronym{AES}{AES}{Advanced Encryption Standard}
\newacronym{BEAST}{BEAST}{Browser Exploit Against SSL/TLS}
\newacronym{CBC}{CBC}{Cipher Block Chaining}
\newacronym{DES}{DES}{Data Encryption Standard}
\newacronym{FIPS}{FIPS}{Federal Information Processing Standards}
\newacronym{HSTS}{HSTS}{HTTP Strict Transport Security}

\newglossaryentry{secret-key encryption}{
  name=secret-key encryption, description={Encryption that uses the
  same key for both encryption and decryption. Also known as
  symmetric-key encryption. Contrast with \gls{public-key encryption}}
  }
\newglossaryentry{symmetric-key encryption}{
  name=symmetric-key encryption, description={See \gls{secret-key
  encryption}} }
\newglossaryentry{block cipher}{
  name=block cipher, description={Symmetric encryption algorithm that
  encrypts and decrypts blocks of fixed size}, }
\newglossaryentry{stream cipher}{
  name=stream cipher, description={Symmetric encryption algorithm that
  encrypts streams of arbitrary size} }
\newglossaryentry{mode of operation}{
  name=mode of operation, description={Generic construction that
  encrypts and decrypts streams, built from a block
  cipher},plural=modes of operation }
\newglossaryentry{ECB mode}{
  name=ECB mode, description={Electronic code book mode; mode of
  operation where plaintext is separated into blocks that are
  encrypted separately under the same key. The default mode in many
  cryptographic libraries, despite many security issues} }
\newglossaryentry{CBC mode}{
  name=CBC mode, description={Cipher block chaining mode; common mode
  of operation where the previous ciphertext block is XORed with the
  plaintext block during encryption} }
\newglossaryentry{CTR mode}{
  name=CTR mode, description={Counter mode; a \gls{nonce} combined
  with a counter produces a sequence of inputs to the block cipher;
  the resulting ciphertext blocks are the keystream} }
\newglossaryentry{nonce}{
  name=nonce, description={\emph{N}umber used \emph{once}. Used in
  many cryptographic protocols. Generally does not have to be secret
  or unpredictable} }
\newglossaryentry{public-key algorithm}{
  name=public-key algorithm, description={Algorithm that uses a pair
  of two related but distinct keys. Also known
  as \glspl{asymmetric-key algorithm}. Examples
  include \gls{public-key encryption} and most \gls{key exchange}
  protocols} }
\newglossaryentry{asymmetric-key algorithm}{
  name=asymmetric-key algorithm, description={See \gls{public-key
  algorithm}} }
\newglossaryentry{public-key encryption}{
  name=public-key encryption, description={Encryption using a pair of
  distinct keys for encryption and decryption. Also known as
  asymmetric-key encryption. Contrast with \gls{secret-key
  encryption}} }
\newglossaryentry{asymmetric-key encryption}{
  name=asymmetric-key encryption, description={See \gls{public-key
  encryption}} }
\newglossaryentry{key exchange}{
  name=key exchange, description={The process of exchanging keys
  across an insecure medium using a particular cryptographic protocol.
  Typically designed to be secure against eavesdroppers. Also known
  as key agreement.} }
\newglossaryentry{key agreement}{
  name=key agreement, description={See \gls{key exchange}}}
\newglossaryentry{oracle}{
  name=oracle, description={A ``black box'' that will perform some
  computation for you.}}
\newglossaryentry{encryption oracle}{
  name=encryption oracle, description={An \gls{oracle} that will
  encrypt some data.}}
