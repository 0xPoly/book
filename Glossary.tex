\newacronym{AEAD}{AEAD}{Authenticated Encryption with Associated Data}
\newacronym{AES}{AES}{Advanced Encryption Standard}
\newacronym{BEAST}{BEAST}{Browser Exploit Against SSL/TLS}
\newacronym{CBC}{CBC}{Cipher Block Chaining}
\newacronym{DES}{DES}{Data Encryption Standard}
\newacronym{FIPS}{FIPS}{Federal Information Processing Standards}
\newacronym{GCM}{GCM}{Galois Counter Mode}
\newacronym{HKDF}{HKDF}{HMAC-based (Extract-and-Expand) Key Derivation Function}
\newacronym{HSTS}{HSTS}{HTTP Strict Transport Security}
\newacronym{IV}{IV}{\gls{initialization vector}}
\newacronym{KDF}{KDF}{key derivation function}
\newacronym{MAC}{MAC}{message authentication code}
\newacronym{PRF}{PRF}{pseudorandom function}
\newacronym{PRP}{PRP}{pseudorandom permutation}

\newglossaryentry{secret-key encryption}{
  name=secret-key encryption, description={Encryption that uses the
  same key for both encryption and decryption. Also known as
  symmetric-key encryption. Contrast with \gls{public-key encryption}}
  }
\newglossaryentry{symmetric-key encryption}{
  name=symmetric-key encryption, description={See \gls{secret-key
  encryption}} }
\newglossaryentry{block cipher}{
  name=block cipher, description={Symmetric encryption algorithm that
  encrypts and decrypts blocks of fixed size}, }
\newglossaryentry{stream cipher}{
  name=stream cipher, description={Symmetric encryption algorithm that
  encrypts streams of arbitrary size} }
\newglossaryentry{mode of operation}{
  name=mode of operation, description={Generic construction that
  encrypts and decrypts streams, built from a block
  cipher},plural=modes of operation }
\newglossaryentry{ECB mode}{
  name=ECB mode, description={Electronic code book mode; mode of
  operation where plaintext is separated into blocks that are
  encrypted separately under the same key. The default mode in many
  cryptographic libraries, despite many security issues} }
\newglossaryentry{CBC mode}{
  name=CBC mode, description={Cipher block chaining mode; common mode
  of operation where the previous ciphertext block is XORed with the
  plaintext block during encryption. Uses an \gls{initialization
  vector} to fill the role of the ``previous'' ciphertext block for
  encrypting the first ciphertext block} }
\newglossaryentry{initialization vector}{
  name=initialization vector, description={Data used to initialize
  some algorithms, e.g. takes the role of the ``block before the first
  block'' in \gls{CBC mode}. Generally not required to be secret, but
  required to be unpredictable. Compare \gls{nonce}, \gls{salt}} }
\newglossaryentry{CTR mode}{
  name=CTR mode, description={Counter mode; a \gls{nonce} combined
  with a counter produces a sequence of inputs to the block cipher;
  the resulting ciphertext blocks are the keystream} }
\newglossaryentry{nonce}{
  name=nonce, description={\emph{N}umber used \emph{once}. Used in
  many cryptographic protocols. Generally does not have to be secret
  or unpredictable, but does have to be unique.
  Compare \gls{initialization vector}, \gls{salt}} }
\newglossaryentry{GCM mode}{
  name=GCM mode, description={Galois Counter Mode; \gls{block
  cipher} \gls{mode of operation} that combines encryption and
  authentication simultaneously. Combination of \gls{CTR mode} with
  a Carter-Wegman MAC.}}
\newglossaryentry{salt}{
  name=salt, description={Random data that is added to a cryptographic
  primitive (usually a one-way function such as a cryptographic hash
  function or a key derivation function) Customizes such functions to
  produce different outputs (provided the salt is different). Can be
  used to prevent e.g. dictionary attacks. Typically does not have to
  be secret, but secrecy may improve security properties of the
  system. Compare \gls{nonce}, \gls{initialization vector}} }
\newglossaryentry{public-key algorithm}{
  name=public-key algorithm, description={Algorithm that uses a pair
  of two related but distinct keys. Also known
  as \glspl{asymmetric-key algorithm}. Examples
  include \gls{public-key encryption} and most \gls{key exchange}
  protocols} }
\newglossaryentry{asymmetric-key algorithm}{
  name=asymmetric-key algorithm, description={See \gls{public-key
  algorithm}} }
\newglossaryentry{public-key encryption}{
  name=public-key encryption, description={Encryption using a pair of
  distinct keys for encryption and decryption. Also known as
  asymmetric-key encryption. Contrast with \gls{secret-key
  encryption}} }
\newglossaryentry{asymmetric-key encryption}{
  name=asymmetric-key encryption, description={See \gls{public-key
  encryption}} }
\newglossaryentry{key exchange}{
  name=key exchange, description={The process of exchanging keys
  across an insecure medium using a particular cryptographic protocol.
  Typically designed to be secure against eavesdroppers. Also known
  as key agreement.} }
\newglossaryentry{key agreement}{
  name=key agreement, description={See \gls{key exchange}}}
\newglossaryentry{oracle}{
  name=oracle, description={A ``black box'' that will perform some
  computation for you}}
\newglossaryentry{encryption oracle}{
  name=encryption oracle, description={An \gls{oracle} that will
  encrypt some data}}
